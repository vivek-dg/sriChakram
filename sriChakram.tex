%\documentclass[preview]{standalone}
%\documentclass[convert={density=300,outext=.png}]{standalone}

\documentclass[convert=pdf2svg]{standalone}
%\documentclass[convert]{standalone}

%\documentclass[]{article}
%\usepackage[top=1.0in, bottom=1.0in, left=0.5in, right=0.5in]{geometry}

%%IF THEN LOGIC
\usepackage{ifthen}
%\usepackage{amsmath}

%%%%%%%%%%%%%%%%%%%%%%%%%%%%%%%%%%
%%%%%%%%%%  FONTS %%%%%%%%%%%%%%%%%%%%

% The fontspec package provides a nice interface to font loading.
\usepackage{fontspec} 
%\setmainfont[Script=Tamil,Scale=0.8]{Latha}
%\setmainfont[Script=Tamil,Scale=0.9]{Arial Unicode MS}

% Set the main font to Nakula (a Devanagari font I found on my system).
%\setmainfont[Script=Devanagari,Scale = 1.2]{Nakula}
%\setmainfont[Script=Devanagari,Scale = 1.2]{Code2000}
\setmainfont[Script=Devanagari,Scale = 1.2]{Chandas}

% Define the \Englishfont command to switch to the 'Fontin' font for English text
\newfontfamily\Englishfont[Script=Latin]{Fontin}
% Define the \eng command as a localized version of \latinfont
\newcommand\eng[1]{{\Englishfont #1}}

%%%%%%%%%%%%%%%%%%%%%%%%%%%%%%%%%
%%%%%%%%%%%   COLOR  %%%%%%%%%%%%
\usepackage[dvipsnames]{xcolor}

%%%%%%%%%%%%%%%%%%%%%%%%%%%%%%%%%%
%%%%%%%%%% TIKZ %%%%%%%%%%%%%%%%%%%%%

\usepackage{tikz}
\usetikzlibrary{shadows}
\usetikzlibrary{fadings}
\usetikzlibrary{calc}
\usetikzlibrary{shapes}
\usetikzlibrary{intersections}


%%%%%%%%%%%
\makeatletter
\newcommand{\gettikzxy}[3]{%
  \tikz@scan@one@point\pgfutil@firstofone#1\relax
  \edef#2{\the\pgf@x}%
  \edef#3{\the\pgf@y}%
}
\makeatother
%%%%%%%%%%%





%%% PETALS !!!
\tikzfading[name=fade out, inner color=transparent!0,
  outer color=transparent!100]

%% Make the petals look more traditional
\def\petal { 
%  [rounded corners=0.9] %
%  (-1,0)%
%  .. controls (-1,0.8) and (-0.2,0.8).. (0,1)%
%  .. controls (0.2,0.8) and (1,0.8).. (1,0)%
%  .. controls (0.7,-1) and (-0.7,-1).. (-1,0)%
  
  %% original from tex
%	[rounded corners=0.5] %
%  	(-1,0)%
%  	.. controls (-1,0.6) and (-0.07,0.8).. (0,1)%
%  	.. controls (0.07,0.8) and (1,0.6).. (1,0)%
%  	.. controls (0.7,-1) and (-0.7,-1).. (-1,0)%
	
%	%% make it wider at the base
%	  [rounded corners=0.5] %
%  		(-1,-0.2)%
%  		.. controls (-1,0.8) and (-0.2,0.6).. (0,1)%
%  		.. controls (0.2,0.6) and (1,0.8).. (1,-0.2)%
	
	%% wider at base but less pointy at top
	
	  [rounded corners=0.5] %
		(-1,-0.2)%
		.. controls (-1,0.8) and (-0.07,0.8).. (0,1)%
		.. controls (0.07,0.8) and (1,0.8).. (1,-0.2)%	
}

%%%%% one layer of petals %%%%%
\def\mainbody{
  \foreach \a in {\af,\as,...,360}
  {
    \begin{scope}[rotate=\a,shift={(0,\ysh)},xscale=\xs,yscale=\ys]
      \draw[color=yellow!90!,thick,fill=\c] %
      [drop shadow={shadow xshift=0.5pt, shadow yshift=-0.5pt}]
      \petal;
    \end{scope}

    % fadings
    \begin{scope}[transform canvas={rotate=\a}, shift={(0,\ysh)},xscale=\xs,yscale=\ys]
      \clip \petal; %
      \fill[path fading=fade out,fill=\fc, opacity=0.7]%
      (0,-0.35) ellipse (1.2 and 0.75);
      \fill[path fading=fade out,fill=\fc, opacity=0.3]%
      (0,-0.2) ellipse (1.2 and 0.4);
      \fill[path fading=fade out,fill=\fc,opacity=0.2] %
      (-0.4,0.6) -- (0,0.9) -- (0.4,0.6);
    \end{scope}
  }
}

%%%%%%%%%%% petal colouring %%%%%%%%%%
% #1 - the color of the petal layer
% #2 - the color of the fading, gets denser closer to the center
% #3 - the x-scale to render the diagram
% #4 - the y-scale to render
% #5 - y shift to be applied
% #6 - increment angle
\def \drawpetals[#1,#2,#3,#4,#5,#6]{
  % ysh - yshift
  % xs - xscale
  % ys - yscale
  % af - first angle in foreach
  % as - second angle in foreach
  % c - color of the petal
  % fc - color of the fading
  \foreach \ysh/\xs/\ys/\af/\as/\c/\fc in {%
    #5/#3/#4/      0/#6/#1!/#2!%
  }
  {
    \mainbody
  }
}

%%%%%%%%%%%%%%%%%%%%%%%%%
%%%%%% Parameters for the Sri Chakra
%%%%%%%%%%%%%%%%%%%%%%%%%

\newcount\mytheta
\def\r{5.3}
\def\n{6} % number of angles
\def\myangles{{0,30,90,150,180,270}} % angles for the 1st south triangle


%%%%%%%%%%%%%%%%%%%%%%%%
%%%% Color for the triangles in SriChakra
%%Bindu color
\def\binduLineColor{Goldenrod}
\def\binduAreaColor{yellow!20!}

%%%Triangles
\def\triangleEdgeColor{Goldenrod}
\def\triangleAreaColor{red!75!}

%%Innermost circle
\def\innerCircleEdgeColor{Goldenrod}
\def\innerCircleAreaColor{White}

%%Inner Petals
\def\innerPetalColor{\triangleAreaColor}
\def\innerPetalGradientColor{\triangleAreaColor}

%%%Middle circle
\def\innerCircleTwoEdgeColor{Red}


%%%Outer Petals
\def\outerPetalColor{\triangleAreaColor}
\def\outerPetalGradientColor{\triangleAreaColor}

%%%Three concentric outer circles
\def\circleEdgeColorOne{Goldenrod}
\def\circleEdgeColorTwo{Goldenrod}
\def\circleEdgeColorThree{Goldenrod}

%%%Radial background
\def\backgroundGradientColorOne{black}
\def\backgroundGradientColorTwo{black}


%%Outer square
%\def\squareLinesColor{\triangleAreaColor}
\def\squareLinesColor{Goldenrod}
\def\squareAreaColor{\backgroundGradientColorOne}

%% Color of OM
\def\omcolor{Goldenrod}

\colorlet{triangle edge}{\triangleEdgeColor}
\colorlet{triangle area}{\triangleAreaColor}




%%%%%%%%%%%%%%%%%%%%%%%%%
%%%%% Color for the triangles in SriChakra
%%%Bindu color
%\def\binduLineColor{green!80!}
%\def\binduAreaColor{yellow!30!}
%
%%%%Triangles
%\def\triangleEdgeColor{yellow!90!}
%\def\triangleAreaColor{red!75!}
%
%%%Innermost circle
%\def\innerCircleEdgeColor{yellow!90!}
%\def\innerCircleAreaColor{White}
%
%%%Inner Petals
%\def\innerPetalColor{\triangleAreaColor}
%\def\innerPetalGradientColor{\triangleAreaColor}
%
%%%%Middle circle
%\def\innerCircleTwoEdgeColor{Red}
%
%
%%%%Outer Petals
%\def\outerPetalColor{\triangleAreaColor}
%\def\outerPetalGradientColor{\triangleAreaColor}
%
%%%%Three concentric outer circles
%\def\circleEdgeColorOne{red!80!}
%\def\circleEdgeColorTwo{orange!80!}
%\def\circleEdgeColorThree{orange!50!}
%
%%%%Radial background
%\def\backgroundGradientColorOne{yellow!30!}
%\def\backgroundGradientColorTwo{white}
%
%
%%%Outer square
%\def\squareLinesColor{\triangleAreaColor}
%\def\squareAreaColor{\backgroundGradientColorOne}
%
%%% Color of OM
%\def\omcolor{Black}
%
%\colorlet{triangle edge}{\triangleEdgeColor}
%\colorlet{triangle area}{\triangleAreaColor}



%%%%%%%%%%%%%%%%%%%%%%%%%
%%%%% Color for the triangles in SriChakra
%%%Bindu color
%\def\binduLineColor{green!80!}
%\def\binduAreaColor{yellow!30!}
%
%%%Triangles
%\def\triangleEdgeColor{Fuchsia}
%\def\triangleAreaColor{Cerulean}
%
%%%Innermost circle
%\def\innerCircleEdgeColor{Red}
%\def\innerCircleAreaColor{White}
%
%%%Inner Petals
%\def\innerPetalColor{ProcessBlue}
%\def\innerPetalGradientColor{\triangleAreaColor}
%
%%%Middle circle
%\def\innerCircleTwoEdgeColor{Red}
%
%%%Outer Petals
%\def\outerPetalColor{BurntOrange}
%\def\outerPetalGradientColor{BurntOrange!80!black!20!}
%
%%%Three concentric outer circles
%%\def\circleEdgeColorOne{red!80!}
%%\def\circleEdgeColorTwo{orange!80!}
%%\def\circleEdgeColorThree{orange!50!}
%
%
%\def\circleEdgeColorOne{ForestGreen}
%\def\circleEdgeColorTwo{ForestGreen}
%\def\circleEdgeColorThree{ForestGreen}
%
%%%Radial background
%\def\backgroundGradientColorOne{white}
%\def\backgroundGradientColorTwo{white}
%
%%%Outer square
%\def\squareLinesColor{Bittersweet}
%\def\squareAreaColor{\backgroundGradientColorOne}
%
%%% Color of OM
%\def\omcolor{Black}
%
%\colorlet{triangle edge}{\triangleEdgeColor}
%\colorlet{triangle area}{\triangleAreaColor}



\tikzset{filled/.style={fill=triangle area, draw=triangle edge,rounded corners=0.5,thick},
    outline/.style={draw=triangle edge,rounded corners=0.5}}

%%%%%%%%%%%%%%%%%%%%%%%
%%%For the outer square rims
\tikzset{ 
	triple/.style args={[#1] in [#2] in [#3]}{
	         #1,preaction={preaction={draw,#3},draw,#2}     
	} 
}

%%%%%%%%%%%%%%%%%%
%%%%%Definitions for outer rim circles
\def\outerradone{1.43*\r} %
\def\outerradtwo{1.41*\r}
\def\outerradthree{1.39*\r}
%%%%% half of diagonal length for the outer rims
\def\distBetLines{0.0286/2 * \r} %% distance between the three lines
\def\rs{(sqrt(2)+0.2)*\outerradone} %% this is the percentage for the half of diagonal for the square rims
\def\ma{0.15} %% this is the gap on the sides of the rectangular rims
\def\ext{0.05} %%% percentage that extends out of the sri chakra in the mouth

%%%%%%%%%%%%%%%%%%%%%%%%%%%%%%%%%%%%%%%




\usetikzlibrary{fpu,decorations.pathreplacing}  

\makeatletter  % Answer to the question 
\def\pgfextractxasmacro#1#2{%  
 \pgf@process{#2}%   
\edef#1{\the\pgf@x}} 
\def\pgfextractyasmacro#1#2{%   
\pgf@process{#2}%   
\edef#1{\the\pgf@y}} 
\def\pgfextractxvecasmacro#1#2#3{%   
% #1 macro where the x coor of the \vec{#2#3} is stored   
% #2 node name   
% #3 node name   
\pgfextractxasmacro{#1}{%     
\pgfpointdiff{\pgfpointanchor{#2}{center}}{\pgfpointanchor{#3}{center}}}} \def\pgfextractyvecasmacro#1#2#3{%   
% #1 macro where the x coor of the \vec{#2#3} is stored   %
% #2 node name   
% #3 node name   
\pgfextractyasmacro{#1}{%     
\pgfpointdiff{
\pgfpointanchor{#2}{center}
}
{\pgfpointanchor{#3}{center}
}
}
}  
\def\pgfgetsineofAOB#1#2#3#4{%   
% #1 macro where the sine of angle AOB is stored   
% #2 node name A  
 % #3 node name O   
% #4 node name B  
 \bgroup   
\pgfkeys{/pgf/fpu,pgf/fpu/output format=fixed}   
\pgfextractxvecasmacro{\pgf@xA}{#3}{#2}%   
\pgfextractyvecasmacro{\pgf@yA}{#3}{#2}%   
\pgfextractxvecasmacro{\pgf@xB}{#3}{#4}%   
\pgfextractyvecasmacro{\pgf@yB}{#3}{#4}%   
\pgfmathparse{%     
((\pgf@xA * \pgf@yB) - (\pgf@xB * \pgf@yA))/(sqrt(\pgf@xA * \pgf@xA     + \pgf@yA * \pgf@yA) * sqrt(\pgf@xB * \pgf@xB + \pgf@yB * \pgf@yB))}%   
\xdef#1{\pgfmathresult}%   
\egroup\ignorespaces}  
\def\pgfgetcosineofAOB#1#2#3#4{%   
% #1 macro where the cosine of angle AOB is stored   
% #2 node name A   
% #3 node name O   
% #4 node name B   
\bgroup   
\pgfkeys{/pgf/fpu,pgf/fpu/output format=fixed}   
\pgfextractxvecasmacro{\pgf@xA}{#3}{#2}%   
\pgfextractyvecasmacro{\pgf@yA}{#3}{#2}%   
\pgfextractxvecasmacro{\pgf@xB}{#3}{#4}%   
\pgfextractyvecasmacro{\pgf@yB}{#3}{#4}%   
\pgfmathparse{%     
((\pgf@xA * \pgf@xB) + (\pgf@yA * \pgf@yB))/(sqrt(\pgf@xA * \pgf@xA     + \pgf@yA * \pgf@yA) * sqrt(\pgf@xB * \pgf@xB + \pgf@yB * \pgf@yB))}%   
\xdef#1{\pgfmathresult}%   
\egroup\ignorespaces}  
\def\pgfgetangleofAOB#1#2#3#4{%   
% #1 macro where the angle AOB is stored   
% #2 node name A   
% #3 node name O   
% #4 node name B   
\bgroup   
\pgfgetsineofAOB{\pgf@sineAOB}{#2}{#3}{#4}%   
\pgfgetcosineofAOB{\pgf@cosineAOB}{#2}{#3}{#4}%   
\pgfmathparse{atan2(\pgf@cosineAOB,\pgf@sineAOB)}%   
\xdef#1{\pgfmathresult}%   
\egroup\ignorespaces} 
% End of the answer 
% Begin mark angle decoration 
\pgfdeclaredecoration{mark angle}{init}{%   
\state{init}[width = 0pt, next state = check for moveto,   persistent precomputation = {%    
 \xdef\pgf@lib@decorations@numofconsecutivelineto{0}}]{}   
\state{check for moveto}[width = 0pt,   
next state=check for lineto,persistent precomputation={%     
\begingroup     
\pgf@lib@decoraions@installinputsegmentpoints
 \ifx\pgfdecorationpreviousinputsegment\pgfdecorationinputsegmentmoveto 
    \gdef\pgf@lib@decorations@numofconsecutivelineto{0}%     
\fi     
\endgroup}]{}   
\state{check for lineto}[width=\pgfdecoratedinputsegmentremainingdistance,    
next state=check for moveto,persistent precomputation={%     
\begingroup     
\pgf@lib@decoraions@installinputsegmentpoints    
 \ifx\pgfdecorationcurrentinputsegment\pgfdecorationinputsegmentlineto
     \xdef\pgf@lib@decorations@numofconsecutivelineto{% 
      \number\numexpr\pgf@lib@decorations@numofconsecutivelineto+1\relax}% 
    \ifcase\pgf@lib@decorations@numofconsecutivelineto\relax     \or     \pgf@process{\pgf@decorate@inputsegment@first}% 
    \xdef\pgf@lib@decorations@first@lineto@point{\the\pgf@x,\the\pgf@y}%  
   \pgf@process{\pgf@decorate@inputsegment@last}%   
  \xdef\pgf@lib@decorations@second@lineto@point{\the\pgf@x,\the\pgf@y}%    
 \pgfmathanglebetweenpoints{\pgf@decorate@inputsegment@last}{%      
 \pgf@decorate@inputsegment@first}%    
 \xdef\pgf@lib@decorations@lineto@startangle{\pgfmathresult}%   
  \or     \pgf@process{\pgf@decorate@inputsegment@last}%    
 \xdef\pgf@lib@decorations@third@lineto@point{\the\pgf@x,\the\pgf@y}%    
 \pgfmathanglebetweenpoints{\pgf@decorate@inputsegment@first}{%      
 \pgf@decorate@inputsegment@last}%    
 \xdef\pgf@lib@decorations@lineto@endangle{\pgfmathresult}%  
   \pgfdecoratedmarkanglecode     \fi     \fi     \endgroup}]{} }  
\pgfqkeys{/pgf/decoration}{%   
mark angle node text/.store in = \pgfdecoratedmarkanglenodetext,   mark angle node text = {},   mark angle code/.store in = \pgfdecoratedmarkanglecode,   mark angle code = {%     
\fill[red,nearly transparent]     (\pgf@lib@decorations@second@lineto@point) --      ($(\pgf@lib@decorations@second@lineto@point)!1cm!     (\pgf@lib@decorations@first@lineto@point)$)      arc(\pgf@lib@decorations@lineto@startangle:     \pgf@lib@decorations@lineto@endangle:1cm) -- cycle;     \node at ($(\pgf@lib@decorations@second@lineto@point) +     ({\pgf@lib@decorations@lineto@startangle +       (\pgf@lib@decorations@lineto@endangle -        \pgf@lib@decorations@lineto@startangle)/2}:1.25cm)$)     {\pgfdecoratedmarkanglenodetext};}}   \makeatletter  \tikzset{mark angle/.style = {%     
postaction = {%       
decorate,       decoration = {mark angle}}}} 


%\coordinate (O) at (0,0);   
%\coordinate (x) at (5,0);   
%\coordinate (y) at (0,5);   
%\coordinate (M) at (30:5);   
%\coordinate (N) at (215:5);  
% \pgfgetangleofAOB{\firstangle}{x}{O}{M}%  
% \pgfgetangleofAOB{\secondangle}{O}{M}{y}%   
%\pgfgetangleofAOB{\thirdangle}{N}{O}{y}% 
%  \draw[mark angle,/pgf/decoration/mark angle node     text={$\firstangle$},red] (x) -- (O) -- (M);   
% \draw[mark angle,/pgf/decoration/mark angle node     text={$\secondangle$},blue] (O) -- (M) -- (y);   
% \draw[mark angle,/pgf/decoration/mark angle node     text={$\thirdangle$},green] (N) -- (O) -- (y);






\begin{document}


\pagestyle {empty} % no page numbers
\pagecolor{black}


%\begin{center}

\begin{tikzpicture}
%radii for the circles
%
\pgfmathparse{(\rs*cos(45))}\let\xxa\pgfmathresult
\pgfmathparse{\rs*sin (45)}\let\yya\pgfmathresult
\pgfmathparse{\rs*cos(135)}\let\xxb\pgfmathresult
\pgfmathparse{\rs*sin (135)}\let\yyb\pgfmathresult
\pgfmathparse{\rs*cos(225)}\let\xxc\pgfmathresult
\pgfmathparse{\rs*sin (225)}\let\yyc\pgfmathresult
\pgfmathparse{\rs*cos(315)}\let\xxd\pgfmathresult
\pgfmathparse{\rs*sin (315)}\let\yyd\pgfmathresult

%%% the points on the outer rim to create the gap for the mouth
%Near zero degree
\pgfmathparse{\rs/sqrt(2)}\let\xxe\pgfmathresult
\pgfmathparse{\ma*\rs}\let\yye\pgfmathresult

%just below 90 degree
\pgfmathparse{\ma*\rs}\let\xxf\pgfmathresult
\pgfmathparse{\rs/sqrt(2)}\let\yyf\pgfmathresult
%Just above 90 degrees
\pgfmathparse{-\ma*\rs}\let\xxg\pgfmathresult
\pgfmathparse{\rs/sqrt(2)}\let\yyg\pgfmathresult
%
%just below 180 degree
\pgfmathparse{-\rs/sqrt(2)}\let\xxh\pgfmathresult
\pgfmathparse{\ma*\rs}\let\yyh\pgfmathresult
%Just above 180 degrees
\pgfmathparse{-\rs/sqrt(2)}\let\xxi\pgfmathresult
\pgfmathparse{-\ma*\rs}\let\yyi\pgfmathresult
%
%just below 270 degree
\pgfmathparse{-\ma*\rs}\let\xxj\pgfmathresult
\pgfmathparse{-\rs/sqrt(2)}\let\yyj\pgfmathresult
%Just above 270 degrees
\pgfmathparse{\ma*\rs}\let\xxk\pgfmathresult
\pgfmathparse{-\rs/sqrt(2)}\let\yyk\pgfmathresult
%
%just below 360 degree
\pgfmathparse{\rs/sqrt(2)}\let\xxl\pgfmathresult
\pgfmathparse{-\ma*\rs}\let\yyl\pgfmathresult


%%% the points on the outer rim to create the effect of the rim moving out
%Near zero degree
\pgfmathparse{(\rs/sqrt(2)) + (\ext*\rs)}\let\xxm\pgfmathresult
\pgfmathparse{\ma*\rs}\let\yym\pgfmathresult

%just below 90 degree
\pgfmathparse{\ma*\rs}\let\xxn\pgfmathresult
\pgfmathparse{(\rs/sqrt(2)) + (\ext*\rs)}\let\yyn\pgfmathresult
%Just above 90 degrees
\pgfmathparse{-\ma*\rs}\let\xxo\pgfmathresult
\pgfmathparse{(\rs/sqrt(2)) + (\ext*\rs)}\let\yyo\pgfmathresult
%
%just below 180 degree
\pgfmathparse{-((\rs/sqrt(2)) + (\ext*\rs))}\let\xxp\pgfmathresult
\pgfmathparse{\ma*\rs}\let\yyp\pgfmathresult
%Just above 180 degrees
\pgfmathparse{-((\rs/sqrt(2)) + (\ext*\rs))}\let\xxq\pgfmathresult
\pgfmathparse{-\ma*\rs}\let\yyq\pgfmathresult
%
%just below 270 degree
\pgfmathparse{-\ma*\rs}\let\xxr\pgfmathresult
\pgfmathparse{-((\rs/sqrt(2)) + (\ext*\rs))}\let\yyr\pgfmathresult
%Just above 270 degrees
\pgfmathparse{\ma*\rs}\let\xxs\pgfmathresult
\pgfmathparse{-((\rs/sqrt(2)) + (\ext*\rs))}\let\yys\pgfmathresult
%
%just below 360 degree
\pgfmathparse{(\rs/sqrt(2)) + (\ext*\rs)}\let\xxt\pgfmathresult
\pgfmathparse{-\ma*\rs}\let\yyt\pgfmathresult


%Find the coordinates
\coordinate (S-1) at (\xxa, \yya); % 45 deg
\coordinate (S-2) at (\xxb, \yyb); % 135 deg
\coordinate (S-3) at (\xxc, \yyc); % 225 deg
\coordinate (S-4) at (\xxd, \yyd); % 315 deg

%Near Mid-points for the main gap
\coordinate (S-5) at (\xxe, \yye); %above zero
\coordinate (S-6) at (\xxf, \yyf);  %below 90
\coordinate (S-7) at (\xxg, \yyg); % above 90
\coordinate (S-8) at (\xxh, \yyh); %below 180
\coordinate (S-9) at (\xxi, \yyi); % above 180
\coordinate (S-10) at (\xxj, \yyj); %below 270
\coordinate (S-11) at (\xxk, \yyk); % above 270
\coordinate (S-12) at (\xxl, \yyl); % below 360

%Near Mid-points for out moving effect for the rim
\coordinate (S-13) at (\xxm, \yym); %above zero
\coordinate (S-14) at (\xxn, \yyn);  %below 90
\coordinate (S-15) at (\xxo, \yyo); % above 90
\coordinate (S-16) at (\xxp, \yyp); %below 180
\coordinate (S-17) at (\xxq, \yyq); % above 180
\coordinate (S-18) at (\xxr, \yyr); %below 270
\coordinate (S-19) at (\xxs, \yys); % above 270
\coordinate (S-20) at (\xxt, \yyt); % below 360

%Determine the intersection points
%First define the line which we will need later
\edef\tripleLineStyle{name path=lineOuter,draw,triple={[line width=\distBetLines*1 cm,\squareLinesColor] in       [line width=\distBetLines*6.5 cm,\squareAreaColor] in       [line width=\distBetLines*8.5 cm,\squareLinesColor]}}



\path\expandafter[\tripleLineStyle] (S-13) -- (S-5) -- (S-1) -- (S-6) -- (S-14);

\path\expandafter[\tripleLineStyle] (S-15) -- (S-7) -- (S-2) -- (S-8) -- (S-16);

\path\expandafter[\tripleLineStyle] (S-17) -- (S-9) -- (S-3) -- (S-10) -- (S-18);

\path\expandafter[\tripleLineStyle] (S-19) -- (S-11) -- (S-4) -- (S-12) -- (S-20);



%%%%%%%%%%%%%%%%%%%%
%%%%%%PUT the letter OM 
%90 degree
\pgfmathparse{0}\let\xom\pgfmathresult
\pgfmathparse{(\rs/sqrt(2))}\let\yom\pgfmathresult
\node [rectangle] (om) at (\xom,\yom){\LARGE \color{\omcolor} ॐ}; 




%%%%%%%%%%%%%%
%% Draw the outer circular rims
%%%%%%%%% Outermost of the outer rims
\shadedraw[shading=radial,outer color=\backgroundGradientColorOne,middle color=\backgroundGradientColorTwo,
        inner color=\backgroundGradientColorTwo,draw=\circleEdgeColorOne,thick,name path=circle1] circle (\outerradone);
 
%\draw[\circleEdgeColorOne,fill=\backgroundGradientColorOne,thick,name path=circle1] circle (\outerradone);

%%%%%%%%% Central ring
\draw[\circleEdgeColorTwo,thick,name path=circle2] circle (\outerradtwo);
%%%%%%%%% Innermost of the outer rims
\draw[\circleEdgeColorThree,thick,name path=circle3] circle (\outerradthree);


%%%%%%%%%%%%%%%%%%%%%%%%%%%%%%%%%
%%%%% CLIPPING - Clip anything outside of the last rim of chakra
%%%%%%%%%%%%%%%%%%%%%%%%%%%%%%%%%
\clip circle (\outerradone);


%%%%%%%%%%%%%%%%%%%%%%%%%%%%%%%%%

%%%%%%%%% Next draw the first layer of outer petals
%%%%%%%%%%% petal colouring %%%%%%%%%%
% #1 - the color of the petal layer
% #2 - the color of the fading, gets denser closer to the center
% #3 - the x-scale to render the diagram
% #4 - the y-scale to render
% #5 - y shift to be applied
% #6 - increment angle

\edef\xscale{0.2 * \r}
\edef\yscale{0.18 * \r}

%%%Traditional petals
%\begin{scope}[even odd rule]
%\clip circle (1.2*\r) circle(\outerradone);  %%
%\drawpetals[\outerPetalColor, \outerPetalGradientColor,\xscale,\yscale,1.2*\r,22.5];  %% outer ring of 16 petals
%\end{scope}
%
%\draw[\circleEdgeColorTwo,thick,name path=circle1] circle (1.2*\r);  %% enclosing circle
%
%\begin{scope}[even odd rule]
%\clip circle (\r) circle(1.2*\r); 
%\drawpetals[\innerPetalColor,\innerPetalGradientColor,\xscale,\yscale,\r,45]; %% inner ring of 8 petals
%\end{scope}

%%%%More recent petals

\begin{scope}[even odd rule]
\clip circle (1.2*\r) circle(\outerradone);  %%
\drawpetals[\outerPetalColor, \outerPetalGradientColor,\xscale*1.175,\yscale,1.2*\r,22.5];  %% outer ring of 16 petals
\end{scope}

\draw[\circleEdgeColorTwo,thick,name path=circle1] circle (1.2*\r);  %% enclosing circle

\begin{scope}[even odd rule]
\clip circle (\r) circle(1.2*\r); 
\drawpetals[\innerPetalColor,\innerPetalGradientColor,\xscale*1.915,\yscale*1.15,0.96*\r,45]; %% inner ring of 8 petals
\end{scope}





%%%Draw the main circle to house the triangles
%\shadedraw[shading=radial,outer color=\backgroundGradientColorOne,middle color=\backgroundGradientColorTwo,
 %       inner color=\backgroundGradientColorTwo,draw=\circleEdgeColorTwo,name path=circle1] circle (\r);
%\draw[orange!80!,fill=yellow!30!,thick,name path=circle1] circle (\r);

%\draw[\innerCircleEdgeColor,fill=\innerCircleAreaColor,name path=circleInner] circle (\r);

\draw[\innerCircleEdgeColor,name path=circleInner] circle (\r);

%define the vertical line thro the center of the circle
\path[name path=vertLine] (0,-\r) -- (0,\r);


%%%%%%%%%%%%%%%%%%%%%%%%%%%%%%%%%
%%%%%%%%%% START OF MAIN CALCULATIONS %%%%%%%%%%

%% First determine the base lines for the first set of triangles in the north and south directions
%%%%%%%%%%%%% SKEW it a bit to get optimal results!!!!
%%%%% SkEWing these angles alters the base of the South 1 triangle
\pgfmathparse{\r*cos(0)}\let\xa\pgfmathresult
\pgfmathparse{\r*sin (0)}\let\ya\pgfmathresult
\pgfmathparse{\r*cos(32.6)}\let\xb\pgfmathresult
\pgfmathparse{\r*sin (32.6)}\let\yb\pgfmathresult
\pgfmathparse{\r*cos(90)}\let\xc\pgfmathresult
\pgfmathparse{\r*sin (90)}\let\yc\pgfmathresult
\pgfmathparse{\r*cos(147.4)}\let\xd\pgfmathresult
\pgfmathparse{\r*sin (147.4)}\let\yd\pgfmathresult
\pgfmathparse{\r*cos(180)}\let\xe\pgfmathresult
\pgfmathparse{\r*sin (180)}\let\ye\pgfmathresult
\pgfmathparse{\r*cos(270)}\let\xf\pgfmathresult
\pgfmathparse{\r*sin (270)}\let\yf\pgfmathresult
%%%%%%%%%%%%% SKEW it a bit to get optimal results!!!!
%%%%% SkEWing these angles alters the base of the North 1 triangle
\pgfmathparse{\r*cos(215.037)}\let\xg\pgfmathresult
\pgfmathparse{\r*sin (215.037)}\let\yg\pgfmathresult
\pgfmathparse{\r*cos(324.963)}\let\xh\pgfmathresult
\pgfmathparse{\r*sin (324.963)}\let\yh\pgfmathresult

%Find the coordinates
\coordinate (A-1) at (\xa, \ya);
\coordinate (A-2) at (\xb, \yb);
\coordinate (A-3) at (\xc, \yc);
\coordinate (A-4) at (\xd, \yd);
\coordinate (A-5) at (\xe, \ye);
\coordinate (A-6) at (\xf, \yf);
\coordinate (A-7) at (\xg, \yg);
\coordinate (A-8) at (\xh, \yh);

%Determine the intersection points
%First define the line which we will need later
\path[name path=line1] (A-1) -- (A-3);
\path[name path=line2] (A-3) -- (A-5);
\path[name path=line3] (A-4) -- (A-6);
\path[name path=line4] (A-6) -- (A-2);

%\draw (A-1) -- (A-3) (A-6) -- (A-2);
%\draw (A-3) -- (A-5) (A-4) -- (A-6);

%Find the intersection points
\path [name intersections={of=line1 and line4}];
\coordinate (i-1) at (intersection-1);
%\fill[fill=green,draw=blue] (i-1) circle (0.8mm); 
\path [name intersections={of=line2 and line3}];
\coordinate (i-2) at (intersection-1);
%\fill[fill=green,draw=blue] (i-2) circle (0.8mm); 

%Draw a line thro the intersection points and extend to meet the circle
%find the half point
\coordinate (south1-m-1) at ($(i-1)!0.5!(i-2)$);
\path[name path=south1-temp] ($(south1-m-1)!1.5!(i-1)$) -- ($(south1-m-1)!1.5!(i-2)$);
%intersection
\path [name intersections={of=south1-temp and circleInner}];
\coordinate (south1-base-1) at (intersection-1);
%\fill[fill=red,draw=blue] (south1-base-1) circle (0.8mm); 
\coordinate (south1-base-2) at (intersection-2);
%\fill[fill=red,draw=blue] (south1-base-2) circle (0.8mm); 

%Now draw the first base triangle facing south 
%%% Draw with color in the end!!!
\path[name path=south1] (A-6) -- (south1-base-1) -- (south1-base-2) -- cycle ;



%%%%%%%%%%%%% 1st South Triangle complete

%%%%%%%%%%%% Next is the first north facing triangle
\path[name path=line5] (A-7) -- (south1-m-1);
\path[name path=line6] (south1-m-1) -- (A-8);

%Find the intersection points
\path [name intersections={of=south1 and line5}];
\coordinate (i-3) at (intersection-2);
%%\fill[fill=green,draw=blue] (i-3) circle (0.8mm); 
\path [name intersections={of=line6 and south1}];
\coordinate (i-4) at (intersection-1);
%%\fill[fill=green,draw=blue] (i-4) circle (0.8mm);


%Draw a line thro the intersection points and extend to meet the circle
%find the half point
\coordinate (north1-m-1) at ($(i-3)!0.5!(i-4)$);
\path[name path=north1-temp] ($(north1-m-1)!2!(i-3)$) -- ($(north1-m-1)!2!(i-4)$);
%intersection
\path [name intersections={of=north1-temp and circleInner}];
\coordinate (north1-base-1) at (intersection-1);
%\fill[fill=red,draw=blue] (north1-base-1) circle (0.8mm); 
\coordinate (north1-base-2) at (intersection-2);
%\fill[fill=red,draw=blue] (north1-base-2) circle (0.8mm);

%Now draw the first base triangle facing south
%%% Draw with color in the end!!!
\path[name path global=north1] (A-3) -- (north1-base-1) -- (north1-base-2) -- cycle;

%%%%%%%%%%%%% 1st North Triangle complete

%%%%%%%%%%%% Next is the seconf south facing triangle

%% with a radius , draw arcs with the base vertices of south 1 triangle to cut the inscribing circle at 2 
%points and find its center. This is the top vertex of the 2nd south facing triangle
 \edef\percR{1.065*\r}  %%%%%%% This is the point of genius
\begin{scope}[shift={(south1-base-1)}] 
\path[name path global=south1-arc-1] (240:\percR) arc (240:270:\percR);
\end{scope}
\begin{scope}[shift={(south1-base-2)}] 
\path[name path global=south1-arc-2] (270:\percR) arc (270:300:\percR);
\end{scope}

%Find the intersection points with circleInner
\path [name intersections={of=circleInner and south1-arc-1}];
\coordinate (i-5) at (intersection-1);
%\fill[fill=green,draw=blue] (i-5) circle (0.8mm); 
\path [name intersections={of=circleInner and south1-arc-2}];
\coordinate (i-6) at (intersection-1);
%\fill[fill=green,draw=blue] (i-6) circle (0.8mm);


%The mid point gives the top vertex
\coordinate (south2-m-1) at ($(i-5)!0.5!(i-6)$);
%\fill[fill=green,draw=blue] (south2-m-1) circle (0.8mm);

%Find the intersection of north triangle 1 with the base of south triangle 1 
\path[name path=south1-base] (south1-base-1) -- (south1-base-2);
%Find the intersection points with north1
\path [name intersections={of=north1 and south1-base}];
\coordinate (i-7) at (intersection-1);
%\fill[fill=green,draw=blue] (i-7) circle (0.8mm); 
\coordinate (i-8) at (intersection-2);
%\fill[fill=green,draw=blue] (i-8) circle (0.8mm);

%Now extend the sides thro these intersection points to a certain distance
% These need  to be stored... the base will be determined after the north 3 triangle is complete
\path[name path=south2-side1-temp] (south2-m-1) -- ($(south2-m-1)!1.5!(i-7)$);
\path[name path=south2-side2-temp] (south2-m-1) -- ($(south2-m-1)!1.5!(i-8)$);


%%%%%%%%%% NORTH 3 Triangle should be completed even before North 2 Triangle
%%%%%%%  It is needed so that we can complete south 2 triangle as well!!!!
% The vertex is the mid point of the south 1 triangle's base


% The sides of SOUTH 2 intersect the north 1 triangle's base at two points. Those intersection points
%% are on the sides of NORTH 3 triangle!!
%Find the intersection of north triangle 1 with the base of south triangle 1 
\path[name path=north1-base] (north1-base-1) -- (north1-base-2);
%Find the intersection points with south 2 sides (we don't have the full south 2 yet!!
\path [name intersections={of=south2-side1-temp and north1-base}];
\coordinate (i-9) at (intersection-1);
%\fill[fill=green,draw=blue] (i-9) circle (0.8mm); 
\path [name intersections={of=south2-side2-temp and north1-base}];
\coordinate (i-10) at (intersection-1);
%\fill[fill=green,draw=blue] (i-10) circle (0.8mm);


%Now extend the sides thro these intersection points to a certain distance
% These will intersect the line that carries the vertex of south 2 triangle at 2 points. 
% Those points will form the base of this triangle
\path[name path=north3-side1-temp] (south1-m-1) -- ($(south1-m-1)!2.5!(i-9)$);
\path[name path=north3-side2-temp] (south1-m-1) -- ($(south1-m-1)!2.5!(i-10)$);

%Find the intersection of these extended lines with the line carrying the vertex of south 2 triangle 
\path[name path=south2-vertex-line] (i-5) -- (i-6);
%Find the intersection points with extended north 3 sides
\path [name intersections={of=south2-vertex-line and north3-side1-temp}];
\coordinate (north3-base-1) at (intersection-1);
%\fill[fill=green,draw=blue] (north3-base-1) circle (0.8mm); 
\path [name intersections={of=south2-vertex-line and north3-side2-temp}];
\coordinate (north3-base-2) at (intersection-1);
%\fill[fill=green,draw=blue] (north3-base-2) circle (0.8mm);

% Draw the North 3 triangle
%%% Draw with color in the end!!!
\path[name path global=north3] (south1-m-1) -- (north3-base-1) -- (north3-base-2) -- cycle;


%%%%%%%%%% Now NORTH 2 TRIANGLE can be drawn
%%%%%%%%%%%%%%%%%%%%%%%%%%5

%% with a radius , draw arcs with the base vertices of north 1 triangle to cut the inscribing circle at 2 
%points and find its center. This is the top vertex of the 2nd south facing triangle
 \edef\percR{1.055*\r}  %%%%%%% This is the point of genius 
\begin{scope}[shift={(north1-base-1)}]         
	\path[name path global=north1-arc-1] (60:\percR) arc (60:90:\percR);
\end{scope}
\begin{scope}[shift={(north1-base-2)}] 
	\path[name path global=north1-arc-2] (90:\percR) arc (90:120:\percR);
\end{scope}

%Find the intersection points with circleInner
\path [name intersections={of=circleInner and north1-arc-1}];
\coordinate (i-11) at (intersection-1);
%\fill[fill=green,draw=blue] (i-11) circle (0.8mm); 
\path [name intersections={of=circleInner and north1-arc-2}];
\coordinate (i-12) at (intersection-1);
%\fill[fill=green,draw=blue] (i-12) circle (0.8mm);


%The mid point gives the top vertex
\coordinate (north2-m-1) at ($(i-11)!0.5!(i-12)$);
%\fill[fill=green,draw=blue] (north2-m-1) circle (0.8mm);

%Find the intersection of south triangle 1 with the base of north triangle 1 
\path[name path=north1-base] (north1-base-1) -- (north1-base-2);
%Find the intersection points with south1
\path [name intersections={of=south1 and north1-base}];
\coordinate (i-13) at (intersection-1);
%\fill[fill=green,draw=blue] (i-13) circle (0.8mm); 
\coordinate (i-14) at (intersection-2);
%\fill[fill=green,draw=blue] (i-14) circle (0.8mm);

%Now extend the sides thro these intersection points to a certain distance
% These need  to be stored... the base will be determined after the north 3 triangle is complete
\path[name path=north2-side1-temp] (north2-m-1) -- ($(north2-m-1)!1.5!(i-13)$);
\path[name path=north2-side2-temp] (north2-m-1) -- ($(north2-m-1)!1.5!(i-14)$);

%%%%%%%%%%% finding the base line for North 2 triangle
% Now find the base points
%
\coordinate (south1-m-2) at ($(south1-base-1)!0.5!(south1-base-2)$);
%
\path[name path=south1-side1](south1-base-2) -- (A-6);
\path[name path=north3-side1](north3-base-1) -- (south1-m-2);
\path [name intersections={of=south1-side1 and north3-side1}];
\coordinate (north2-tempbase-1) at (intersection-1);
%\fill[fill=green,draw=blue] (north2-tempbase-1) circle (0.8mm);
%
\path[name path=south1-side2](south1-base-1) -- (A-6);
\path[name path=north3-side2](north3-base-2) -- (south1-m-2); 
\path [name intersections={of=south1-side2 and north3-side2}];
\coordinate (north2-tempbase-2) at (intersection-1);
%\fill[fill=green,draw=blue] (north2-tempbase-2) circle (0.8mm);
%
\coordinate (north2-m-2) at ($(north2-tempbase-1)!0.5!(north2-tempbase-2)$);
%\fill[fill=green,draw=blue] (north2-m-2) circle (0.8mm);
\path[name path=north2-tempbaseline] ($(north2-m-2)!2!(north2-tempbase-1)$) -- ($(north2-m-2)!2!(north2-tempbase-2)$);

%%%%%%%%%%%%%%%%%%%%%%%
% Now find the base points for north 2
\path [name intersections={of=north2-tempbaseline and north2-side2-temp}];
\coordinate (north2-base-1) at (intersection-1);
%\fill[fill=green,draw=blue] (north2-base-1) circle (0.8mm); 
\path [name intersections={of=north2-tempbaseline and north2-side1-temp}];
\coordinate (north2-base-2) at (intersection-1);
%\fill[fill=green,draw=blue] (north2-base-2) circle (0.8mm);

% Draw the North 2 triangle
%%% Draw with color in the end!!!
\path[name path global=north2] (north2-m-1) -- (north2-base-1) -- (north2-base-2) -- cycle;


%%%%%%%%%%%% Back to completing the SOUTH 2 Triangle
%%%%% Now find the points of intersection between the sides of North 2 triangle with 
%%%%% the base of South 1 triangle. 

%Find the intersection of north triangle 2 with the base of south triangle 1 
\path [name intersections={of=north2 and south1-base}];
\coordinate (i-15) at (intersection-1);
%\fill[fill=green,draw=blue] (i-15) circle (0.8mm); 
\coordinate (i-16) at (intersection-2);
%\fill[fill=green,draw=blue] (i-16) circle (0.8mm);

% Now draw diagonals between these points and the points where  North 2 triangle intersects with 
%%%% the base of North 1 triangle. 
\path[name path=north2-base-south1-base-D1](i-13) -- (i-15);
\path[name path=north2-base-south1-base-D2](i-14) -- (i-16);

%%%% These diagonals intersect the south 2 triangle at two points.
%%% Find and mark them
\path [name intersections={of=north2-base-south1-base-D1 and south2-side2-temp}];
\coordinate (north4-base-1) at (intersection-1);
%\fill[fill=green,draw=blue] (north4-base-1) circle (0.8mm); 
\path [name intersections={of=north2-base-south1-base-D2 and south2-side1-temp}];
\coordinate (north4-base-2) at (intersection-1);
%\fill[fill=green,draw=blue] (north4-base-2) circle (0.8mm);

%%%  ONLY after South 3 triangle is finished we can move forward
%The mid point gives the top vertex for SOUTH 3 triange
\coordinate (south3-m-1) at ($(north4-base-1)!0.5!(north4-base-2)$);
%\fill[fill=green,draw=blue] (south3-m-1) circle (0.8mm);

%Now extend the sides thro these intersection points of North 2  triangle with South 1 base
% These will intersect the line that carries the vertex of north 2 triangle at 2 points. 
% Those points will form the base of this triangle
\path[name path=south3-side1-temp] (south3-m-1) -- ($(south3-m-1)!2.5!(i-15)$);
\path[name path=south3-side2-temp] (south3-m-1) -- ($(south3-m-1)!2.5!(i-16)$);

%Find the intersection of these extended lines with the line carrying the vertex of north 2 triangle 
\path[name path=north2-vertex-line] (i-11) -- (i-12);
%Find the intersection points with extended south 3 sides
\path [name intersections={of=north2-vertex-line and south3-side1-temp}];
\coordinate (south3-base-1) at (intersection-1);
%\fill[fill=green,draw=blue] (south3-base-1) circle (0.8mm); 
\path [name intersections={of=north2-vertex-line and south3-side2-temp}];
\coordinate (south3-base-2) at (intersection-1);
%\fill[fill=green,draw=blue] (south3-base-2) circle (0.8mm);

% Draw the South 3 triangle
%%% Draw with color in the end!!!
\path[name path global=south3] (south3-m-1) -- (south3-base-1) -- (south3-base-2) -- cycle;


%%%%%%%%%%% Back to finding the base line for South 2 triangle
% Now find the base points
\path[name path=north1-side1](north1-base-1) -- (A-3);
\path[name path=south3-side1](south3-base-1) -- (south3-m-1);
\path [name intersections={of=north1-side1 and south3-side1-temp}];
\coordinate (south2-tempbase-1) at (intersection-1);
%\fill[fill=green,draw=blue] (south2-tempbase-1) circle (0.8mm);
%
\path[name path=north1-side2](north1-base-2) -- (A-3);
\path[name path=south3-side2](south3-base-2) -- (south3-m-1); 
\path [name intersections={of=north1-side2 and south3-side2-temp}];
\coordinate (south2-tempbase-2) at (intersection-1);
%\fill[fill=green,draw=blue] (south2-tempbase-2) circle (0.8mm);
%
\coordinate (south2-m-2) at ($(south2-tempbase-1)!0.5!(south2-tempbase-2)$);
%\fill[fill=green,draw=blue] (south2-m-2) circle (0.8mm);
\path[name path=south2-tempbaseline] ($(south2-m-2)!2!(south2-tempbase-1)$) -- ($(south2-m-2)!2!(south2-tempbase-2)$);


%% now find the itersection of this line with the temporary sides of south 2 triangle and figure out the base vertices
% Now find the base points for south 2
\path [name intersections={of=south2-tempbaseline and south2-side2-temp}];
\coordinate (south2-base-1) at (intersection-1);
%\fill[fill=green,draw=blue] (south2-base-1) circle (0.8mm); 
\path [name intersections={of=south2-tempbaseline and south2-side1-temp}];
\coordinate (south2-base-2) at (intersection-1);
%\fill[fill=green,draw=blue] (south2-base-2) circle (0.8mm);

% Draw the South 2 triangle
%%% Draw with color in the end!!!
\path[name path global=south2] (south2-m-1) -- (south2-base-1) -- (south2-base-2) -- cycle;


%%%%%%%%%% North 4 triangle 
%% Now North 4 triangle's vertex is the mid point of the base for south 2 triangle
%The mid point gives the top vertex
\coordinate (north4-m-1) at ($(south2-base-1)!0.5!(south2-base-2)$);
%\fill[fill=green,draw=blue] (north4-m-1) circle (0.8mm);

% Draw the North 4 triangle
%%% Draw with color in the end!!!
\path[name path global=north4] (south2-m-2) -- (north4-base-1) -- (north4-base-2) -- cycle;

%%%%%%%%5 For the south 4 triangle
%%%%%% Find the intersections of the sides of North 4 and South 3 triangles
%%%55 Join them and extend them to meet the sides of North 2 triangle

\path [name intersections={of=north4 and south3}];
\coordinate (south4-tempbase-1) at (intersection-1);
%\fill[fill=green,draw=blue] (south4-tempbase-1) circle (0.8mm); 
\coordinate (south4-tempbase-2) at (intersection-3);
%\fill[fill=green,draw=blue] (south4-tempbase-2) circle (0.8mm);

%The mid point on the base
\coordinate (south4-m-2) at ($(south4-tempbase-1)!0.5!(south4-tempbase-2)$);
%\fill[fill=green,draw=blue] (south4-m-2) circle (0.8mm);

%Now extend the sides thro these intersection points of North 2  triangle with South 1 base
% These will intersect the line that carries the vertex of north 2 triangle at 2 points. 
% Those points will form the base of this triangle
\path[name path=south4-tempbaseline] ($(south4-m-2)!2.5!(south4-tempbase-1)$) -- ($(south4-m-2)!2.5!(south4-tempbase-2)$);

%%% Find the intersection points with North 2 triangle
\path [name intersections={of=north2 and south4-tempbaseline}];
\coordinate (south4-base-1) at (intersection-1);
%\fill[fill=green,draw=blue] (south4-base-1) circle (0.8mm); 
\coordinate (south4-base-2) at (intersection-2);
%\fill[fill=green,draw=blue] (south4-base-2) circle (0.8mm);

%% The mid point of the North triangle 2's base points is the vertex
\coordinate (north2-m-2) at ($(north2-base-1)!0.5!(north2-base-2)$);
%\fill[fill=green,draw=blue] (north2-m-2) circle (0.8mm);


%%%Draw the South 4 triangle
%%% Draw with color in the end!!!
\path[name path global=south4] (north2-m-2) -- (south4-base-1) -- (south4-base-2) -- cycle;




%%%%%%%%5 For the south 5 triangle
%%%%%% Find the intersections of the sides of North 3 and South 3 triangles
%%%55 Join them and extend them to meet the sides of North 4 triangle

\path [name intersections={of=north3 and south3}];
\coordinate (south5-tempbase-1) at (intersection-1);
%\fill[fill=green,draw=blue] (south5-tempbase-1) circle (0.8mm); 
\coordinate (south5-tempbase-2) at (intersection-2);
%\fill[fill=green,draw=blue] (south5-tempbase-2) circle (0.8mm);

%The mid point on the base
\coordinate (south5-m-2) at ($(south5-tempbase-1)!0.5!(south5-tempbase-2)$);
%\fill[fill=green,draw=blue] (south5-m-2) circle (0.8mm);

%Now extend the sides thro these intersection points of North 4  triangle 
% Those points will form the base of this triangle
\path[name path=south5-tempbaseline] ($(south5-m-2)!2.5!(south5-tempbase-1)$) -- ($(south5-m-2)!2.5!(south5-tempbase-2)$);

%%% Find the intersection points with North 4 triangle
\path [name intersections={of=north4 and south5-tempbaseline}];
\coordinate (south5-base-1) at (intersection-1);
%\fill[fill=green,draw=blue] (south5-base-1) circle (0.8mm); 
\coordinate (south5-base-2) at (intersection-2);
%\fill[fill=green,draw=blue] (south5-base-2) circle (0.8mm);

%% The mid point of the North triangle 1's base points is the vertex
\coordinate (north1-m-2) at ($(north1-base-1)!0.5!(north1-base-2)$);
%\fill[fill=green,draw=blue] (north1-m-2) circle (0.8mm);


%%%Draw the South 5 triangle
%%% Draw with color in the end!!!
\path[name path global=south5] (north1-m-2) -- (south5-base-1) -- (south5-base-2) -- cycle;

%%%%%%%%%%%%%%%%%%%%%%%%%%%%%%%%%%%
%%%%%%% END OF CALCULATIONS %%%%%%%%%%%%%%%

%color the intersection of both the triangles to be white
\begin{scope}
	\draw[filled, even odd rule]
%	\shadedraw[shading=radial,outer color=ForestGreen,middle color=ForestGreen,
%        inner color=Cyan,draw=black,name path=chakram, even odd rule]%
	%south1
	(A-6) -- (south1-base-1) -- (south1-base-2) -- cycle
	%north1
	(A-3) -- (north1-base-1) -- (north1-base-2) -- cycle
	%north3
	 (south1-m-1) -- (north3-base-1) -- (north3-base-2) -- cycle
	%north2
	(north2-m-1) -- (north2-base-1) -- (north2-base-2) -- cycle
	%south3
	 (south3-m-1) -- (south3-base-1) -- (south3-base-2) -- cycle
	%south2
	 (south2-m-1) -- (south2-base-1) -- (south2-base-2) -- cycle
	%north4
	(south2-m-2) -- (north4-base-1) -- (north4-base-2) -- cycle
	%south4
	(north2-m-2) -- (south4-base-1) -- (south4-base-2) -- cycle
	%south5
	(north1-m-2) -- (south5-base-1) -- (south5-base-2) -- cycle; 
	\node[anchor=south] at (current bounding box.north) {};
\end{scope}

%\draw[fill=white] (A-6) -- (south1-base-1) -- (south1-base-2) -- cycle;

%%%%%%%%%%%%%%%%%%%%%%%%%
%%%%%%%%%%% Finally draw the bindu
\draw[\binduLineColor,fill=\binduAreaColor,thick,name path=circleBindu] circle (\r/20);
%mark the center
\fill[fill=red] (0,0) circle (\r/100);


%%%%%%%%%%%%% Now 
%\pgfgetangleofAOB{\firstangle}{south3-base-1}{south3-m-1}{south3-base-2}%  
%\draw[mark angle,/pgf/decoration/mark angle node     text={$\firstangle$},black] (south3-base-1) -- (south3-m-1) -- (south3-base-2);  
%
%\pgfgetangleofAOB{\secondangle}{south3-m-1}{south3-base-2}{south3-base-1}%  
%\draw[mark angle,/pgf/decoration/mark angle node     text={$\secondangle$},black] (south3-m-1) -- (south3-base-2) --  (south3-base-1);  
%
%\pgfgetangleofAOB{\thirdangle}{south3-m-1}{south3-base-1}{south3-base-2}%  
%\draw[mark angle,/pgf/decoration/mark angle node     text={$\thirdangle$},black] (south3-m-1) --  (south3-base-1) -- (south3-base-2) ;  

%\pgfgetangleofAOB{\thirdangle}{south5-base-1}{north1-m-2}{south5-base-2}%  
%\draw[mark angle,/pgf/decoration/mark angle node     text={$\thirdangle$},black]  (south5-base-1) -- (north1-m-2) -- (south5-base-2) ;  

%\pgfgetangleofAOB{\thirdangle}{north1-m-2}{south5-base-1}{south5-base-2}%  
%\draw[mark angle,/pgf/decoration/mark angle node     text={$\thirdangle$},black]   (north1-m-2) -- (south5-base-1) -- (south5-base-2) ;  

\end{tikzpicture}


%\end{center}

\eng{

% now find the angle of the inner most triangle i.e. angle 

}

\end{document}